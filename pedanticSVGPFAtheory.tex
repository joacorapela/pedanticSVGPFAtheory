\documentclass[12pt]{article}

\usepackage{apalike}
\let\bibhang\relax
\usepackage{natbib}
\usepackage{amsmath}
\usepackage{url}
\usepackage{hyperref}
\usepackage[left=1.0cm,right=1cm,top=3cm,bottom=3cm]{geometry}
\usepackage{amsfonts,amssymb}
\usepackage{mathtools}

\def\labelitemi{--}

\title{Pedantic description of the svGPFA theory}

\author{Joaqu\'{i}n Rapela}

\begin{document}

\maketitle

\abstract{Here we provide a pedantic description of the sparse variational
Gaussian process factor analysis (svGPFA) theory, expanding on
\cite{dunckerAndSahani18,dunckerAndSahani18supplementary}.}

\section{GPFA model}

\begin{equation}
    \begin{aligned}
        p(\{x_{kr}(\cdot)\}_{k=1,r=1}^{K,R}) &= \prod_{r=1}^R\prod_{k=1}^Kp(x_{kr}(\cdot))\\
        x_{kr}(\cdot) &\sim \mathcal{GP}(\mu_k(\cdot),\kappa_k(\cdot,\cdot))&&{\text{for}\; k=1, \ldots, K\;\text{and}\;r=1,\ldots, R}\\
        h_{nr}(\cdot) &= \sum_{k=1}^Kc_{nk}x_{kr}(\cdot) + d_n&&{\text{for}\; n=1, \ldots, N\;\text{and}\;r=1,\ldots, R}\\
        p(\{y_{nr}\}_{n=1,r=1}^{N,R}|\{h_{nr}(\cdot)\}_{n=1,r=1}^{N,R}) &= \prod_{r=1}^R\prod_{n=1}^Np(y_{nr}|h_{nr}(\cdot))
    \end{aligned}
    \label{eq:gpfaModel}
\end{equation}

\noindent where $x_{kr}(\cdot)$ is the latent process $k$ in trial $r$, $h_{nr}(\cdot)$ is the embedding process for neuron $n$ and trial $r$ and $y_{nr}$ is the activity of neuron $n$ in trial $r$.

Notes:

\begin{itemize}

    \item the first equation shows that the latent processes are independent,

    \item the second equation shows that the latent processes share mean and
        covariance functions across trials. That is, for any $k$, the mean and
        covariance functions  of latents processes of different trials,
        $x_{kr}(\cdot), r=1,\ldots, R$, are the same ($\mu_k(\cdot)$ and
        $\kappa_k(\cdot,\cdot)$),

    \item the fourth equation shows that, given the embedding processes, the
        responses of different neurons are independent.

\end{itemize}

\section{GPFA with inducing points model}

To use the sparse variational framework for Gaussian processes,
\cite{dunckerAndSahani18} augmented the GPFA model by introducing inducing
points $u_{kr}$ for each latent process $k$ and trial $r$. The inducing points
$\mathbf{u}_{kr}$ represent evaluations of the latent process $x_{kr}(\cdot)$
at locations $\mathbf{z}_{kr}=\left[z_{kr}[0],\ldots,z_{kr}[M_{kr}-1]\right]$.
A joint prior over the latent process $x_{kr}(\cdot)$ and its inducing points
$\mathbf{u}_{kr}$ is given in Eq.~\ref{eq:gpfaWithIndPointsPrior}.

\begin{equation}
    \begin{aligned}
        p(\{x_{kr}(\cdot)\}_{k=1,r=1}^{K,R},\{\mathbf{u}_{kr}\}_{k=1,r=1}^{K,R}) &= p(\{x_{kr}(\cdot)\}_{k=1,r=1}^{K,R}|\{\mathbf{u}_{kr}\}_{k=1,r=1}^{K,R})p(\{\mathbf{u}_{kr}\}_  {k=1,r=1}^{K,R})\\
        p(\{x_{kr}(\cdot)\}_{k=1,r=1}^{K,R}|\{\mathbf{u}_{kr}\}_  {k=1,r=1}^{K,R}) &= \prod_{k=1}^k\prod_{r=1}^{R}p(x_{kr}(\cdot)|\mathbf{u}_{kr})\\
        p(\{\mathbf{u}_{kr}\}_{k=1,r=1}^{K,R})&=\prod_{k=1}^k\prod_{r=1}^{R}p(\mathbf{u}_{kr})\\
        p(\mathbf{u}_{kr})&=\mathcal{N}(\mathbf{0},K^{kr}_{zz})
    \end{aligned}
    \label{eq:gpfaWithIndPointsPrior}
\end{equation}

\noindent where $K_{zz}^{(kr)}[i,j]=\kappa_k(z_{kr}[i],z_{kr}[j])$.

We next derive the functional form of $p(x_{kr}(\cdot)|\mathbf{u}_{kr})$.

Define the random vector $\mathbf{x}_{kr}$ as the random process
$x_{kr}(\cdot)$ evaluated at times
$\mathbf{t}^{(r)}\coloneqq\left\{t_1^{(r)},\ldots,t_M^{(r)}\right\}$ (i.e.,
$\mathbf{x}_{kr}\coloneqq[x_{kr}(t_1^{(r)}),\ldots,x_{kr}(t_M^{(r)})]^\intercal$).
Because the inducing points $\mathbf{u}_{kr}$ are evaluations of the
latent process $x_{kr}(\cdot)$ at $\mathbf{z}_{kr}$, then $\mathbf{x}_{kr}$ and
$\mathbf{u}_{kr}$ are jointly Gaussian:

\begin{equation}
    p\left(\left[\begin{array}{c}
        \mathbf{u}_{kr}\\
        \mathbf{x}_{kr}
    \end{array}\right]\right)
    =\mathcal{N}\left(\left.\left[\begin{array}{c}
        \mathbf{u}_{kr}\\
        \mathbf{x}_{kr}
    \end{array}\right]\right|\left[\begin{array}{c}
        \mathbf{0}\\
        \mathbf{0}
    \end{array}\right],\left[\begin{array}{cc}
        K_\mathbf{zz}^{(kr)}&K_\mathbf{zt}^{(kr)}\\
        K_\mathbf{tz}^{(kr)}&K_\mathbf{tt}^{(r)}
    \end{array}\right]\right)
    \label{eq:prior}
\end{equation}

\noindent where
$K_{tz}^{(kr)}[i,j]=\kappa_k(t^{(r)}_i,z_{kr}[j])$,
$K_{zt}^{(kr)}[i,j]=\kappa_k(z_{kr}[i],t_j^{(r)})$
and
$K_{tt}^{(r)}[i,j]=\kappa_k(t_i^{(r)},t_j^{(r)})$.

Now, applying the formula for the conditional pdf for jointly Normal random
vectors~\citep[][Eq. 2.116]{bishop06} to Eq.~\ref{eq:prior}, we obtain

\begin{equation}
    p(\mathbf{x}_{kr}|\mathbf{u}_{kr})=\mathcal{N}\left(\mathbf{x}_{kr}\left|K_\mathbf{tz}^{(kr)}\left(K_{zz}^{(kr)}\right)^{-1}\mathbf{u}_{kr},\;K_\mathbf{tt}^{(r)}-K_\mathbf{tz}^{(kr)}\left(K_{zz}^{(kr)}\right)^{-1}K_\mathbf{zt}^{(kr)}\right.\right)
    \label{eq:latentConditionalIndPointsVector}
\end{equation}

Because Eq.~\ref{eq:latentConditionalIndPointsVector} is valid for any
$\mathbf{t}^{(r)}$, it follows that

\begin{equation*}
    p(x_{kr}(\cdot)|\mathbf{u}_{kr})=\mathcal{GP}\left(\tilde{\mu}_{kr}(\cdot), \tilde{\kappa}_{kr}(\cdot,\cdot\right))
\end{equation*}

\noindent with

\begin{equation*}
    \begin{aligned}
        \tilde{\mu}_{kr}(t)&=\kappa_k(t,\mathbf{z}_{kr})\left(K_{zz}^{(kr)}\right)^{-1}\mathbf{u}_{kr},\\
        \tilde{\kappa}_k(t,t')&=\kappa_k(t,t')-\kappa_k(t,\mathbf{z}_{kr})\left(K_{zz}^{(kr)}\right)^{-1}\kappa_k(\mathbf{z}_{kr},t')
    \end{aligned}
\end{equation*}

\noindent which is Eq.~3 in \citet{dunckerAndSahani18}.

\section{Derivation of the svGPFA variational lower bound}

To derive the sparse variational lower bound we have the freedom of using
$\{x_{kr}(\cdot)\}_{k=1,r=1}^{K,R}$ or $\{h_{nr}(\cdot)\}_{n=1,r=1}^{N,R}$ as
latent processes in the complete-data likelihood\footnote{To simplify notation below we write
$\{x_{kr}(\cdot)\}_{k=1,r=1}^{K,R}$ as $\{x_{kr}(\cdot)\}$ and similarly for $\{\mathbf{u}_{kr}\}_{k=1,r=1}^{K,R}$ and $\{h_{nr}(\cdot)\}_{n=1,r=1}^{N,R}$}. In
\cite{dunckerAndSahani18supplementary} the authors used the former, which
requires the non-trivial proof of Eq.~\ref{eq:difficultProofDS18}.

\begin{align}
    \text{E}_{q(\{x_{kr}(\cdot)\})}\log p(\{\mathbf{y}_{nr}\}|\{x_{kr}(\cdot)\})=\text{E}_{q(\{h_{nr}(\cdot)\})}\log p(\{\mathbf{y}_{nr}\}|\{h_{nr}(\cdot)\})
    \label{eq:difficultProofDS18}
\end{align}

\noindent Here we use the latter, which avoids the need of Eq.~\ref{eq:difficultProofDS18}. Then the complete data likelihood is given in Eq.~\ref{eq:completeDataLikelihood} and a variational lower bound in Eq.~\ref{eq:lowerBound}.

\begin{align}
    p(\{\mathbf{y}_{nr}\}, \{h_{nr}(\cdot)\}, \{\mathbf{u}_{kr}\})=p(\{\mathbf{y}_{nr}\}|\{h_{nr}(\cdot)\})p(\{h_{nr}(\cdot)\}|\{\mathbf{u}_{kr}\})p(\{\mathbf{u}_{kr}\})
    \label{eq:completeDataLikelihood}
\end{align}

\begin{align}
    \log p(\{\mathbf{y}_{nr}\})\ge E_{q(\{h_{nr}(\cdot)\}, \{\mathbf{u}_{kr}\})}\log\left(p(\{\mathbf{y}_{nr}\}|\{h_{nr}(\cdot)\}, \{\mathbf{u}_{kr}\}\right)-KL\left\{q(\{h_{nr}(\cdot)\},\{\mathbf{u}_{kr}\})||p(\{h_{nr}(\cdot)\},\{\mathbf{u}_{kr}\})\right\}
    % \log p(\{\mathbf{y}_{nr}\})\ge\int\int q(\{h_{nr}(\cdot)\}, \{\mathbf{u}_{kr}\})\log\left(\frac{p(\{\mathbf{y}_{nr}\}, \{h_{nr}(\cdot)\}, \{\mathbf{u}_{kr}\})}{q(\{h_{nr}(\cdot)\}, \{\mathbf{u}_{kr}\})}\right)d\{\mathbf{u}_{kr}\}d\{h_{nr}(\cdot)\}
    \label{eq:lowerBound}
\end{align}

As in \cite{dunckerAndSahani18supplementary}, we choose the approximating variational distribution in Eq.~\ref{eq:approxDistr}.

\begin{equation}
    \begin{aligned}
        q(\{x_{kr}(\cdot)\}, \{\mathbf{u}_{kr}\})&=q(\{x_{kr}(\cdot)\}|\{\mathbf{u}_{kr}\})q(\{\mathbf{u}_{kr}\})\\
        q(\{x_{kr}(\cdot)\}|\{\mathbf{u}_{kr}\})&=p(\{x_{kr}(\cdot)\}|\{\mathbf{u}_{kr}\})\\
        q(\{\mathbf{u}_{kr}\})&=\prod_{r=1}^R\prod_{k=1}^Kq(\mathbf{u}_{kr})\\
        q(\mathbf{u}_{kr})&=\mathcal{N}(\mathbf{u}_{kr}|\mathbf{m}_{kr},S_{kr})
    \end{aligned}
    \label{eq:approxDistr}
\end{equation}

Because $h_{nr}(\cdot)$ is a function of $\{x_{kr}(\cdot)\}$ only (Eq.~\ref{eq:gpfaModel}), and because $q(\{x_{kr}(\cdot)\}|\{\mathbf{u}_{kr}\})=p(\{x_{kr}(\cdot)\}|\{\mathbf{u}_{kr}\})$ (Eq.~\ref{eq:approxDistr}), then 

\begin{align}
    q(\{h_{nr}(\cdot)\}|\{\mathbf{u}_{kr}\})=p(\{h_{nr}(\cdot)\}|\{\mathbf{u}_{kr}\})
    \label{eq:qH_equals_pH}
\end{align}

% \pagebreak
The first term in Eq.~\ref{eq:lowerBound} can be rewritten as

\begin{equation}
    \begin{aligned}
        &E_{q(\{h_{nr}(\cdot)\},\{\mathbf{u}_{kr}\})}\log\left(p(\{\mathbf{y}_{nr}\}|\{h_{nr}(\cdot)\},\{\mathbf{u}_{kr}\}\right)=\\
        &\int\int q(\{h_{nr}(\cdot)\},\{\mathbf{u}_{kr}\})\log\left(p(\{\mathbf{y}_{nr}\}|\{h_{nr}(\cdot)\},\{\mathbf{u}_{kr}\}\right)d\{\mathbf{u}_{kr}\}d\{h_{nr}(\cdot)\}=\\
        &\int\int q(\{h_{nr}(\cdot)\},\{\mathbf{u}_{kr}\})\log\left(p(\{\mathbf{y}_{nr}\}|\{h_{nr}(\cdot)\}\right)d\{\mathbf{u}_{kr}\}d\{h_{nr}(\cdot)\}=\\
        &\int\left(\int q(\{h_{nr}(\cdot)\},\{\mathbf{u}_{kr}\})d\{\mathbf{u}_{kr}\}\right)\log\left(p(\{\mathbf{y}_{nr}\}|\{h_{nr}(\cdot)\}\right)d\{h_{nr}(\cdot)\}=\\
        &\int q(\{h_{nr}(\cdot)\})\log\left(p(\{\mathbf{y}_{nr}\}|\{h_{nr}(\cdot)\}\right)d\{h_{nr}(\cdot)\}=\\
        &\int q(\{h_{nr}(\cdot)\})\log\left(\prod_{r=1}^R\prod_{n=1}^Np(\mathbf{y}_{nr}|h_{nr}(\cdot))\right)d\{h_{nr}(\cdot)\}=\\
        &\sum_{r=1}^R\sum_{n=1}^N\int q(\{h_{nr}(\cdot)\})\log\left(p(\mathbf{y}_{nr}|h_{nr}(\cdot))\right)d\{h_{nr}(\cdot)\}=\\
        &\sum_{r=1}^R\sum_{n=1}^N\int \int q(\{h_{n'r'}(\cdot)\}_{n'\ne n, r'\ne r}|h_{nr}(\cdot))q(h_{nr}(\cdot))\log\left(p(\mathbf{y}_{nr}|h_{nr}(\cdot))\right)d\{h_{n'r'}(\cdot)\}_{n'\ne n, r'\ne r}dh_{nr}(\cdot)=\\
        &\sum_{r=1}^R\sum_{n=1}^N\int\left(\int q(\{h_{n'r'}(\cdot)\}_{n'\ne n, r'\ne r}|h_{nr}(\cdot))d\{h_{n'r'}(\cdot)\}_{n'\ne n,r'\ne r}\right)q(h_{nr}(\cdot))\log\left(p(\mathbf{y}_{nr}|h_{nr}(\cdot))\right)dh_{nr}(\cdot)=\\
        &\sum_{r=1}^R\sum_{n=1}^N\int q(h_{nr}(\cdot))\log\left(p(\mathbf{y}_{nr}|h_{nr}(\cdot))\right)dh_{nr}(\cdot)=\\
        &\sum_{r=1}^R\sum_{n=1}^NE_{q(h_{nr}(\cdot))}\log\left(p(\mathbf{y}_{nr}|h_{nr}(\cdot))\right)
    \end{aligned}
   \label{eq:expectedLogLike}
\end{equation}

Notes:

\begin{itemize}

    \item the second line follows from the previous one because, given the
        embedding process, activities of neurons are independent of inducing
        points (Eq.~\ref{eq:completeDataLikelihood}),

    \item the fifth line follows from the previous one because, given embedding
        processes, responses of neurons are independent of each other (last
        line in Eq.~\ref{eq:gpfaModel}),

\end{itemize}

\pagebreak
The second term in Eq.~\ref{eq:lowerBound} can be rewritten as

\begin{equation}
    \begin{aligned}
        KL\left\{q(\{h_{nr}(\cdot)\},\{\mathbf{u}_{kr}\})||p(\{h_{nr}(\cdot)\},\{\mathbf{u}_{kr}\})\right\}&=\int\int q(\{h_{nr}(\cdot)\},\{\mathbf{u}_{kr}\})\log\frac{q(\{h_{nr}(\cdot)\},\{\mathbf{u}_{kr}\})}{p(\{h_{nr}(\cdot)\},\{\mathbf{u}_{kr}\})}d\{h_{nr}(\cdot)\}d\{\mathbf{u}_{kr}\}\\
        &=\int\int q(\{h_{nr}(\cdot)\},\{\mathbf{u}_{kr}\})\log\frac{q(\{h_{nr}(\cdot)\}|\{\mathbf{u}_{kr}\})q(\{\mathbf{u}_{kr}\})}{p(\{h_{nr}(\cdot)\}|\{\mathbf{u}_{kr}\})p(\{\mathbf{u}_{kr}\})}d\{h_{nr}(\cdot)\}d\{\mathbf{u}_{kr}\}\\
        &=\int\int q(\{h_{nr}(\cdot)\},\{\mathbf{u}_{kr}\})\log\frac{q(\{\mathbf{u}_{kr}\})}{p(\{\mathbf{u}_{kr}\})}d\{h_{nr}(\cdot)\}d\{\mathbf{u}_{kr}\}\\
        &=\int\int q(\{h_{nr}(\cdot)\},\{\mathbf{u}_{kr}\})d\{h_{nr}(\cdot)\}\log\frac{q(\{\mathbf{u}_{kr}\})}{p(\{\mathbf{u}_{kr}\})}d\{\mathbf{u}_{kr}\}\\
        &=\int q(\{\mathbf{u}_{kr}\})\log\frac{q(\{\mathbf{u}_{kr}\})}{p(\{\mathbf{u}_{kr}\})}d\{\mathbf{u}_{kr}\}\\
        &=\int q(\{\mathbf{u}_{kr}\})\log\frac{\prod_{r=1}^R\prod_{k=1}^Kq(\mathbf{u}_{kr})}{\prod_{r=1}^R\prod_{k=1}^Kp(\mathbf{u}_{kr})}d\{\mathbf{u}_{kr}\}\\
        &=\int q(\{\mathbf{u}_{kr}\})\sum_{r=1}^R\sum_{k=1}^K\log\frac{q(\mathbf{u}_{kr})}{p(\mathbf{u}_{kr})}d\{\mathbf{u}_{kr}\}\\
        &=\sum_{r=1}^R\sum_{k=1}^K\int q(\{\mathbf{u}_{kr}\})\log\frac{q(\mathbf{u}_{kr})}{p(\mathbf{u}_{kr})}d\{\mathbf{u}_{kr}\}\\
        &=\sum_{r=1}^R\sum_{k=1}^K\int\prod_{k'=1}^K\prod_{r'=1}^Rq(\mathbf{u}_{k'r'})\log\frac{q(\mathbf{u}_{kr})}{p(\mathbf{u}_{kr})}d\{\mathbf{u}_{kr}\}\\
        &=\sum_{r=1}^R\sum_{k=1}^K\int q(\mathbf{u}_{kr})\log\frac{q(\mathbf{u}_{kr})}{p(\mathbf{u}_{kr})}d\mathbf{u}_{kr}\\
        &=\sum_{r=1}^R\sum_{k=1}^K KL(q(\mathbf{u}_{kr})||p(\mathbf{u}_{kr}))
    \end{aligned}
    \label{eq:kl}
\end{equation}

Notes:

\begin{itemize}

    \item the third line follows from the previous one by Eq.~\ref{eq:qH_equals_pH},

    \item the tenth line follows from the previous because, for $k'\neq k$ or
        $r'\neq r$, the factors $q(\mathbf{u}_{k'r'})$ integrate out to one.

\end{itemize}

Replacing Eq.~\ref{eq:expectedLogLike} and Eq.~\ref{eq:kl} into Eq.~\ref{eq:lowerBound} we obtain

\begin{align}
    \log p(\{\mathbf{y}_{nr}\})\ge \sum_{r=1}^R\sum_{n=1}^NE_{q(h_{nr}(\cdot))}\log\left(p(\mathbf{y}_{nr}|h_{nr}(\cdot))\right)-\sum_{r=1}^R\sum_{k=1}^K KL(q(\mathbf{u}_{kr})||p(\mathbf{u}_{kr}))
    \label{eq:lowerBoundFinal}
\end{align}

\noindent which is Eq.~4 in \cite{dunckerAndSahani18}.

\section{Variational distribution of $h_{nr}(\cdot)$}

For the calculation of the lower bound in the right-hand side of
Eq.~\ref{eq:lowerBoundFinal}, below we derive the distribution
$q(h_{nr}(\cdot))$.

We first deduce the distribution $q(x_{xr}(\cdot))$. Note, from
Eq.~\ref{eq:gpfaWithIndPointsPrior}, that for any $P\in\mathbb{N}$ and for any
$\mathbf{t}=(t_1,\ldots,t_P)\in\mathbb{R}^P$ the approximate variational
posterior of the random vectors
$\mathbf{x}_{kr}=(x_{kr}(t_1),\ldots,x_{kr}(t_P))$ and $\mathbf{u}_{kr}$ is
jointly Gaussian

\begin{equation}
    \begin{aligned}
        q(\mathbf{x}_{kr},\mathbf{u}_{kr})&=p(\mathbf{x}_{kr}|\mathbf{u}_{kr})q(\mathbf{u}_{kr})\\
                                          &=\mathcal{N}\left(\mathbf{x}_{kr}|K_{tz}^{kr}(K_{zz}^{kr})^{-1}\mathbf{u}_{kr},\;K_{tt}^k-K_{tz}^{kr}(K_{zz}^{kr})^{-1}K_{zt}^{kr}\right)\mathcal{N}(\mathbf{u}_{kr}|\mathbf{m}_{kr},\;S_{kr})
    \end{aligned}
    \label{eq:qxu}
\end{equation}

\noindent where $K_{tt}$,$K_{tz}$,$K_{zt}$, and $K_{zz}$ are covariance matrices obtained by evalating of $\kappa_k(t,t')$,$\kappa_k(t,z)$,$\kappa_k(z,t)$, and $\kappa_k(z,z')$, respectively, at $t,t'\in \{t_1,\ldots t_P\}$ and $z,z'\in \{\mathbf{z}_{kr}[1],\ldots,\mathbf{z}_{kr}[M_{kr}]\}$. Next, using the expression for the marginal of a joint Gaussian distribution (e.g., Eq.~2.115 in \cite{bishop06}) we obtain

\begin{align}
    q(\mathbf{x}_{kr})=\mathcal{N}\left(\mathbf{x}_{kr}|K_{tz}^{kr}(K_{zz}^{kr})^{-1}\mathbf{m}_{kr},\;K_{tt}^k+K_{tz}^{kr}\left((K_{zz}^{kr})^{-1}S_{kr}(K_{zz}^{kr})^{-1}-(K_{zz}^{kr})^{-1  }\right)K_{zt}^{kr}\right)
    \label{eq:qxRandomVec}
\end{align}

Because Eq.~\ref{eq:qxRandomVec} holds for any $P\in\mathbb{N}$ and for any $(t_1,\ldots,t_P)\in\mathbb{R}^P$ then

\begin{equation}
    \begin{aligned}
        q(x_{kr}(\cdot))&=\mathcal{GP}\left(\breve\mu_{kr}(\cdot),\breve\kappa_{kr}(\cdot,\cdot)\right)\\
        \breve\mu_{kr}(t)&=\kappa_k(t,z_{kr})(K_{zz}^{kr})^{-1}\mathbf{m}_{kr},\\
        \breve\kappa_{kr}(t,t')&=\kappa_k(t,t')+\kappa_k(t,z_{kr})\left((K_{zz}^{kr})^{-1}S_{kr}(K_{zz}^{kr})^{-1}-(K_{zz}^{kr})^{-1}\right)\kappa_k(z_{kr},t')
    \end{aligned}
    \label{eq:qx}
\end{equation}

Finally, because affine trasformations of Gaussians are Gaussians,
$h_{nr}(\cdot)$ is an affine transformation of $\{x_{kr}(\cdot)\}$ (which are
Gaussians, Eq.~\ref{eq:qx}), then the approximate posterior of $h_{nr}(\cdot)$
is the Gaussian process in Eq.~\ref{eq:qh}.

\begin{equation}
    \begin{aligned}
        q(h_{nr}(\cdot))&=\mathcal{GP}\left(\tilde\mu_{nr}(\cdot),\tilde\kappa_{nr}(\cdot,\cdot)\right)\\
        \tilde\mu_{nr}(t)&=\sum_{k=1}^Kc_{nk}\breve\mu_{kr}(t)\\
        \tilde\kappa_{nr}(t,t')&=\sum_{k=1}^Kc_{nk}^2\breve\kappa_{kr}(t,t')
    \end{aligned}
    \label{eq:qh}
\end{equation}

\noindent which is Eq.~5 in \cite{dunckerAndSahani18}.

\bibliographystyle{apalike}
\bibliography{latentsVariablesModels,machineLearning}
\end{document}
